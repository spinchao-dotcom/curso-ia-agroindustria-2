\documentclass[11pt,a4paper]{article}
\usepackage[utf8]{inputenc}
\usepackage[spanish]{babel}
\usepackage{geometry}
\usepackage{booktabs}
\usepackage{listings}
\usepackage{xcolor}

\geometry{margin=2.5cm}

\definecolor{codebg}{RGB}{245,247,250}

\lstset{
    backgroundcolor=\color{codebg},
    basicstyle=\ttfamily\small,
    breaklines=true,
    frame=single
}

\title{Reporte Taller 02.3\\Vectorización con NumPy}
\author{Tu Nombre}
\date{\today}

\begin{document}

\maketitle

\section{Introducción}

Este reporte presenta los resultados del análisis de humedad de una finca durante 365 días en 100 zonas diferentes, utilizando operaciones vectorizadas con NumPy.

\section{Metodología}

Se procesó una matriz de dimensiones $365 \times 100$ que representa mediciones diarias de humedad del suelo en porcentaje. Las operaciones realizadas fueron:

\begin{itemize}
    \item Detección de zonas en sequía (humedad < 30\%)
    \item Aplicación automática de riego
    \item Clasificación espacial de zonas
\end{itemize}

\section{Resultados}

\subsection{Benchmark de Rendimiento}

% TODO: Completa con los resultados de benchmark.py

\begin{table}[h]
\centering
\begin{tabular}{lc}
\toprule
Método & Tiempo (s) \\
\midrule
Loop Python & [COMPLETAR] \\
NumPy vectorizado & [COMPLETAR] \\
\textbf{Speedup} & [COMPLETAR]x \\
\bottomrule
\end{tabular}
\caption{Comparación de rendimiento}
\end{table}

\subsection{Detección de Sequía}

% TODO: Completa con los resultados de vectorizacion.py

\begin{itemize}
    \item Celdas en sequía: [COMPLETAR]
    \item Porcentaje del área: [COMPLETAR]\%
\end{itemize}

\subsection{Clasificación de Zonas}

% TODO: Completa con los resultados de clasificar_zonas()

\begin{table}[h]
\centering
\begin{tabular}{lcc}
\toprule
Clasificación & Celdas & Porcentaje \\
\midrule
SEQUIA & [COMPLETAR] & [COMPLETAR]\% \\
NORMAL & [COMPLETAR] & [COMPLETAR]\% \\
EXCESO & [COMPLETAR] & [COMPLETAR]\% \\
\bottomrule
\end{tabular}
\caption{Distribución de zonas por humedad}
\end{table}

\section{Conclusiones}

La vectorización con NumPy demostró ser significativamente más eficiente que el procesamiento con bucles Python, logrando un speedup de [COMPLETAR]x. Este enfoque es fundamental para el procesamiento de grandes volúmenes de datos agrícolas.

\end{document}
