\documentclass[aspectratio=169,11pt]{beamer}

% Beamer base (estructura de frames, títulos, etc.) [web:168]
\usetheme{Madrid} % Tema común y legible [web:168]
\usecolortheme{beaver}

\usepackage[spanish]{babel}
\usepackage{graphicx}
\usepackage{hyperref}
\usepackage{xcolor}
\usepackage{listings}

% =========================
% Estilo de código (bash)
% =========================
\definecolor{terminalbg}{RGB}{28, 28, 28}
\definecolor{terminalfg}{RGB}{135, 255, 0}

\lstdefinestyle{bashstyle}{
  backgroundcolor=\color{terminalbg},
  basicstyle=\ttfamily\scriptsize\color{terminalfg},
  breaklines=true,
  language=bash,
  showstringspaces=false,
  keywordstyle=\color{cyan},
  commentstyle=\color{gray!70}\itshape
}

% =========================
% Metadata
% =========================
\title[Semana 1]{Semana 1: El Entorno del Ingeniero}
\subtitle{Linux \,|\, Bash \,|\, Pipes \,|\, awk \,|\, Git}
\author{Curso IA Agroindustria}
\date{Enero 2026}
\institute{Repositorio: \texttt{curso-ia-agroindustria-2}}

\begin{document}

% =========================
% Portada
% =========================
\begin{frame}
  \titlepage
\end{frame}

% =========================
% Agenda
% =========================
\begin{frame}[fragile]{Agenda}
\begin{itemize}
\item Objetivo de la semana y criterio de éxito.
\item Terminal: navegar, organizar proyectos, inspeccionar datos.
\item Pipes y \texttt{awk}: mini-pipelines para CSV.
\item Git: bitácora científica (commits + .gitignore).
\item Entregable A1: script reproducible + repo en GitHub.
\end{itemize}
\end{frame}

% =========================
% Objetivo
% =========================
\begin{frame}[fragile]{Objetivo (en 1 frase)}
Al final de la semana podrás tomar un dataset en CSV, inspeccionarlo, validar calidad básica, automatizar un reporte y versionarlo con Git.
\end{frame}

% =========================
% Estructura proyecto
% =========================
\begin{frame}[fragile]{Estructura mínima de proyecto}
\begin{block}{Convención recomendada}
\begin{itemize}
\item \texttt{data/raw}: datos originales (no se versionan).
\item \texttt{data/processed}: datos listos para análisis.
\item \texttt{scripts}: automatización (bash/python).
\item \texttt{reports}: salidas (txt, tablas, figuras).
\end{itemize}
\end{block}

\begin{lstlisting}[style=bashstyle]
mkdir -p proyecto_ia/{data/raw,data/processed,scripts,reports}
\end{lstlisting}
\end{frame}

% =========================
% Terminal esencial
% =========================
\begin{frame}[fragile]{Terminal esencial (mínimo viable)}
\begin{columns}
\column{0.55\textwidth}
\begin{itemize}
\item \texttt{pwd}, \texttt{ls -lh}, \texttt{cd}
\item \texttt{mkdir -p}, \texttt{cp}, \texttt{mv}, \texttt{rm -i}
\item \texttt{find}, \texttt{du -sh}, \texttt{df -h}
\end{itemize}

\column{0.45\textwidth}
\begin{lstlisting}[style=bashstyle]
pwd
ls -lh
cd proyecto_ia
mkdir -p data/raw
find . -name "*.csv"
du -sh data/*
\end{lstlisting}
\end{columns}
\end{frame}

% =========================
% Inspección CSV
% =========================
\begin{frame}[fragile]{Inspección rápida de CSV}
\begin{itemize}
\item Ver encabezado y muestra: \texttt{head}.
\item Ver últimas filas: \texttt{tail}.
\item Contar registros: \texttt{wc -l}.
\item Navegar sin abrir Excel: \texttt{less}.
\end{itemize}

\begin{lstlisting}[style=bashstyle]
head -n 5 data/raw/sensores.csv
tail -n 5 data/raw/sensores.csv
wc -l data/raw/sensores.csv
less data/raw/sensores.csv
\end{lstlisting}
\end{frame}

% =========================
% Calidad de datos
% =========================
\begin{frame}[fragile]{Señales de mala calidad (rápidas)}
\begin{itemize}
\item Faltantes típicos: \texttt{,,}
\item Codificación rara / caracteres no ASCII.
\item Columnas inconsistentes (separador incorrecto).
\end{itemize}

\begin{lstlisting}[style=bashstyle]
grep ",," data/raw/sensores.csv | head
grep -P "[^\x00-\x7F]" data/raw/sensores.csv | head
\end{lstlisting}
\end{frame}

% =========================
% Pipes
% =========================
\begin{frame}[fragile]{Pipes: entrada \textrightarrow proceso \textrightarrow salida}
\begin{itemize}
\item El pipe (\texttt{|}) conecta comandos: salida de A \(\rightarrow\) entrada de B.
\item Construye pipelines sencillos y potentes sin abrir Python.
\end{itemize}

\begin{lstlisting}[style=bashstyle]
# Frecuencia por sensor (col 1)
cut -d',' -f1 data/raw/sensores.csv | sort | uniq -c | sort -nr | head
\end{lstlisting}
\end{frame}

% =========================
% awk
% =========================
\begin{frame}[fragile]{awk: estadísticas rápidas en columnas}
\begin{lstlisting}[style=bashstyle]
# Promedio temperatura (col 3)
awk -F',' '{s+=$3} END {print "Temp_prom:", s/NR}' data/raw/sensores.csv

# Max humedad (col 4) ignorando header
awk -F',' 'NR>1 {if($4>max) max=$4} END {print "Hum_max:", max}' \
  data/raw/sensores.csv
\end{lstlisting}

\begin{block}{Idea}
\texttt{awk} es ideal para validaciones rápidas antes de cargar datos en pandas.
\end{block}
\end{frame}

% =========================
% Git básico
% =========================
\begin{frame}[fragile]{Git como bitácora científica}
\begin{itemize}
\item Un commit es un “fotograma” del proyecto.
\item Commits pequeños + mensajes claros = reproducibilidad y auditoría.
\item \texttt{.gitignore} evita subir datos brutos y basura.
\end{itemize}

\begin{lstlisting}[style=bashstyle]
git init
git add .
git commit -m "feat: estructura inicial"
git status
git log --oneline --graph
\end{lstlisting}
\end{frame}

% =========================
% .gitignore
% =========================
\begin{frame}[fragile]{.gitignore mínimo para la semana 1}
\begin{lstlisting}[style=bashstyle]
cat > .gitignore << EOF
data/raw/
__pycache__/
.ipynb_checkpoints/
*.log
*.tmp
EOF
\end{lstlisting}
\end{frame}

% =========================
% Entregable A1
% =========================
\begin{frame}[fragile]{Entregable A1 (qué se entrega)}
\begin{enumerate}
\item Repo con estructura \texttt{data/ scripts/ reports/}.
\item \texttt{scripts/setup\_proyecto.sh} idempotente.
\item \texttt{reports/validacion\_inicial.txt} generado por el script.
\item Al menos 3 commits con mensajes tipo \texttt{feat:/docs:/fix:}.
\end{enumerate}
\end{frame}

% =========================
% Checklist final
% =========================
\begin{frame}[fragile]{Checklist (si esto funciona, pasaste la semana)}
\begin{itemize}
\item Puedes navegar y organizar proyectos solo con terminal.
\item Puedes inspeccionar CSV y detectar problemas básicos.
\item Puedes construir un pipeline con \texttt{cut | sort | uniq} o \texttt{awk}.
\item Puedes crear commits claros y usar \texttt{.gitignore}.
\end{itemize}
\end{frame}

% =========================
% Cierre
% =========================
\begin{frame}[fragile]{Siguiente semana}
Semana 2: NumPy para alto rendimiento (vectorización, profiling, memoria).
\end{frame}

\end{document}
